\chapter{Abstract}
\label{ch:abstract}

This experiment specially helps to understand the use of MATLAB for solving communication engineering problem.
By the help of this experiment we can understand how the data converts from digital to analog and the convert again
from analog to digital. We can also understand the sampling theorem and the effect of sampling rate on the signal.
And how the receiver device get back the original signal by the help of sampling and we can see it visually by graph.
This experiment also helps to develop the basic knowledge of MATLAB environment, commands and syntax.

\chapter{Introduction}
\label{ch:intro}

%-------------------------------------------------------

The proprietary multi-paradigm programming language and numerical computing environment known as MATLAB were created
by MathWorks. MATLAB is an acronym for "MATrix LABoratory." Matrix manipulation, function and data visualization,
algorithm implementation, user interface building, and connecting with other programming languages are all possible
with MATLAB.
    {\vskip 1em}
We can use MATLAB for
\begin{itemize}
    \item Analyze data
    \item Develop algorithms
    \item Create models and applications
\end{itemize}
MATLAB lets you take your ideas from research to production by deploying to enterprise applications and embedded
devices, as well as integrating with Simulink® and "Model-Based Design".
{\vskip 1em}
An array that doesn't need to be dimensioned is the fundamental data element in the interactive system known as
Matlab. This saves the user time compared to writing programs in a variety of technical computer tasks, particularly
those involving matrix and vector operations. scalar, inert programming languages like C or Fortran.
    {\vskip 1em}
Toolboxes are a kind of application-specific solutions offered by Matlab. Being extremelyIt's crucial for the majority
of Matlab users that toolboxes enable learning and application of specializedtechnology. The so-called M toolboxes
are extensive collections of Matlab routines.files that enhance the Matlab environment to address certain issue
types.

%%%%%%%%%%%%%%%%%%%%%%%%%%%%%%%%%%%%%%%%%%%%%%%%%%%%%
%%% SUBSECTION NAME
%%%%%%%%%%%%%%%%%%%%%%%%%%%%%%%%%%%%%%%%%%%%%%%%%%%%%

\chapter{Performance Task}
\label{sec:task}
\raggedright{Calculation :} \\
{\vskip 1em}
Here, My ID is $20-43658-2$\\
The main equation is $x = A cos(2\pi ft)$ \\
so we can write $x_1(t) = A_1 cos(2\pi f_1t)$ and,\\
we can also write $x_2(t) = A_2 cos(2\pi f_2t)$ \\
Now, \\
$f_1 = CX10 = 4X10 = 40$ \\
$f_2 = FX10 = 5X10 = 50$ \\
$A_1 = GD = 8X3 = 24$ \\
$A_2 = AXF = 2X5 = 10$ \\
Sample-per-period = 20 \\
$f_s$ = FX(Sample-per-period) = $50X20 = 1000$ \\
Here, there is several frequencies, in that case, we should consider the highest frequency
so that we can get the maximum amount of sampling to get better wave signal. \\
$X_3 = X_1 + X_2$

{\vskip 1em}
\raggedright{Code :}
\begin{verbatim}
    %{
    My ID = 20-43658-2
            AB-CDEFG-H
    %}

    %{ Setting up variables %}
    f1 = 4*10
    f2 = 5*10
    A1 = 8*3
    A2 = 2*5
    fs = 50*20
    t  = (0:(1/fs):1)

    %{ Calculating x1 x2 & x3 %}
    x1 = A1*cos(2*pi*f1*t)
    x2 = A2*cos(2*pi*f2*t)
    x3 = x1 + x2

    %{ Plotting x1 x2 & x3 %}
    subplot(4,1,1)
    plot(t,x1,'g')
    xlabel('Time')
    ylabel('Amplitude')
    title('X1')

    subplot(4,1,2)
    plot(t,x2)
    xlabel('Time')
    ylabel('Amplitude')
    title('X2')

    subplot(4,1,3)
    plot(t,x3,'r')
    xlabel('Time')
    ylabel('Amplitude')
    title('X3')

    %{ Plotting Frequency Domain Graph of x3 %}
    length = length(x3)
    length_in_power = 2^nextpow2(length)

    fx3 = fft(x3,length_in_power)
    ffx3 = fx3(1:length_in_power/2)
    xfft = fs*(0:length_in_power/2-1)/length_in_power

    subplot(4,1,4)
    plot(xfft,abs(ffx3),'k')
    xlabel('Amplitude')
    ylabel('Frequency')
    title('Frequency Domain Graph of X3')
\end{verbatim}
% 
% Figure 1: Task-2
% 
\begin{figure}[htbp]
    \begin{center}
        \fbox{\includegraphics*[width=15cm]{temp.png}}
        \caption{Performance Task}
        \label{fig:task2}
    \end{center}
\end{figure}

\chapter{Discussion}
As it was an important experiment on the basic of matlab so we  have  taken the values in different variables.
We breakdown in an atomic level and then we have to calculate the values of the variables for more clarification.
We have calculate both calculator and the MATLAB so the possibility of error is zero.

\chapter{Conclusion}
As the name suggests MATLAB is a matrix laboratory. It is a high-level language and interactive environment that enables
us to perform computationally intensive tasks faster than with traditional programming languages such as C, C++, and Fortran.
MATLAB allows matrix manipulations, plotting of functions and data, implementation of algorithms, creation of user interfaces,
and interfacing with programs written in other languages, including C, C++, C\#, Java, Fortran and Python.